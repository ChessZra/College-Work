\documentclass[11pt]{article}
\usepackage{amsmath, amsfonts, amsthm, amssymb}  % Some math symbols
\usepackage{lmodern}  % A modern version of LaTeX's famous default font. 
\usepackage{microtype}
\usepackage{fullpage}
\usepackage[x11names, rgb]{xcolor}
\usepackage{graphicx}
\usepackage{tikz}
\usetikzlibrary{decorations,arrows,shapes,automata,positioning}
\tikzset{
->, % makes the edges directed
% >=stealth’, % makes the arrow heads bold
node distance=1cm, % specifies the minimum distance between two nodes. Change if necessary.
every state/.style={thick, fill=gray!10}, % sets the properties for each ’state’ node
initial text=$ $, % sets the text that appears on the start arrow
auto
}

\usepackage{etoolbox}
\usepackage{enumerate}
\usepackage{listings}

\setlength{\parindent}{0pt}
\setlength{\parskip}{5pt plus 1pt}

\newcommand{\nopagenumbers}{
    \pagestyle{empty}
}

\def\indented#1{\list{}{}\item[]}
\let\indented=\endlist

\providetoggle{questionnumbers}
\settoggle{questionnumbers}{true}
\newcommand{\noquestionnumbers}{
    \settoggle{questionnumbers}{false}
}

\newcounter{questionCounter}
\newenvironment{question}[2][\arabic{questionCounter}]{%
    \addtocounter{questionCounter}{1}%
    \setcounter{partCounter}{0}%
    \vspace{.25in} \hrule \vspace{0.4em}%
        \noindent{\bf \iftoggle{questionnumbers}{#1: }{}#2}%
    \vspace{0.8em} \hrule \vspace{.10in}%
}{$ $\newpage}

\newcounter{partCounter}[questionCounter]
\renewenvironment{part}[1][\alph{partCounter}]{%
    \addtocounter{partCounter}{1}%
    \vspace{.10in}%
    \begin{indented}%
       {\bf (#1)} %
}{\end{indented}}

\def\show#1{\ifdefempty{#1}{}{#1\\}}




%%%%%%%%%%%%%%%%% Identifying Information %%%%%%%%%%%%%%%
%%		 		For 301, we'd rather you DIDN'T tell us who you are   		%%
%%				in your homework so that we're not biased when we     		%%
%% 				grade. So, even if you fill this information in, it   			%%
%% 				will not show up in the document unless you uncomment 		%%
%%				 \show\myname and \show\myemail below                  		%%
%%%%%%%%%%%%%%%%%%%%%%%%%%%%%%%%%%%%%%%%%%%
\newcommand{\myhwname}{Homework 6}
\newcommand{\myname}{Anon}
\newcommand{\myemail}{anon@uic.edu}
%%%%%%%%%%%%%%%%%%%%%%%%%%%%%%%%%%%%%%%%%%%%%%%%%%%%%%%%%%%
\newcommand{\header}{%
\begin{center}
    {\Large \show\myhwname}
    % \show\myname
    % \show\myemail
    \today
\end{center}}
%%%%%%%%%%%%%%%%%%% Document Options %%%%%%%%%%%%%%%%%%%%%%
% \noquestionnumbers
\nopagenumbers
%%%%%%%%%%%%%%%%%%%%%%%%%%%%%%%%%%%%%%%%%%%%%%%%%%%%%%%%%%%

\begin{document}
\header

\begin{question}{I Think I Recognize This One}
    Suppose, for the sake of contradiction, that $TR_{TM}$ is decidable. Therefore a TM E decides it. \\
    A = "On input $<M, w>$, \\
    1. Construct M' = "On input x, \\
    \hspace*{35px} 1. If the input is not in the correct format, reject. \\
    \hspace*{35px} 2. Run M on w, if M accepts w, accept. Otherwise, reject. \\
    2. Simulate E with M' as an input. If E accepts, accept. Otherwise, reject." \\
    A accepts exactly when M accepts w and rejects otherwise. This is a contradiction because we know that $A_{TM}$ is undecidable. Thus, $TR_{TM}$ must be undecidable.
\end{question}

\begin{question}{A LENGTH-y Decision}
    Suppose, for the sake of contradiction, that $LENGTH_{TM}$ is decidable. Therefore a TM E decides it. \\
    A = "On input $<M, w>$, \\
    1. Construct M' = "On input x, \\
    \hspace*{35px} 1. If the input doesn't have a length, reject. \\
    \hspace*{35px} 2. Run M on w, if M accepts w, accept. Otherwise, reject. \\
    2. Simulate E with M' as an input. If E accepts, accept. Otherwise, reject." \\
    A accepts exactly when M accepts w and rejects otherwise. This is a contradiction because we know that $A_{TM}$ is undecidable. Thus, $LENGTH_{TM}$ must be undecidable.
\end{question}

\begin{question}{Sort This One Out}
    M = "On input $<L,K>$, \\
    1. We will initialize a variable that keeps track of the maximum element. \\
    2. Iterate through each element in L. \\
    3. For every element in L, if the element is greater than the maximum element, update the max element to that. \\
    4. Once you reach the end of the list, return the max." \\

    The loop only iterates one through the list so the time complexity is O(N). Therefore, M runs in polynomial time relative to the size of L.
\end{question}
\end{document}
