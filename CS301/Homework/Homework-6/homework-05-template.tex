\documentclass[11pt]{article}
\usepackage{amsmath, amsfonts, amsthm, amssymb}  % Some math symbols
\usepackage{lmodern}  % A modern version of LaTeX's famous default font. 
\usepackage{microtype}
\usepackage{fullpage}

\usepackage[x11names, rgb]{xcolor}
\usepackage{graphicx}
\usepackage{tikz}
\usepackage{pdfpages}
\graphicspath{ {./img/} }
\usetikzlibrary{decorations,arrows,shapes,automata,positioning}
\tikzset{
->, % makes the edges directed
% >=stealth’, % makes the arrow heads bold
node distance=1cm, % specifies the minimum distance between two nodes. Change if necessary.
every state/.style={thick, fill=gray!10}, % sets the properties for each ’state’ node
initial text=$ $, % sets the text that appears on the start arrow
auto
}

\usepackage{etoolbox}
\usepackage{enumerate}
\usepackage{listings}

\setlength{\parindent}{0pt}
\setlength{\parskip}{5pt plus 1pt}

\newcommand{\nopagenumbers}{
    \pagestyle{empty}
}

\def\indented#1{\list{}{}\item[]}
\let\indented=\endlist

\providetoggle{questionnumbers}
\settoggle{questionnumbers}{true}
\newcommand{\noquestionnumbers}{
    \settoggle{questionnumbers}{false}
}

\newcounter{questionCounter}
\newenvironment{question}[2][\arabic{questionCounter}]{%
    \addtocounter{questionCounter}{1}%
    \setcounter{partCounter}{0}%
    \vspace{.25in} \hrule \vspace{0.4em}%
        \noindent{\bf \iftoggle{questionnumbers}{#1: }{}#2}%
    \vspace{0.8em} \hrule \vspace{.10in}%
}{$ $\newpage}

\newcounter{partCounter}[questionCounter]
\renewenvironment{part}[1][\alph{partCounter}]{%
    \addtocounter{partCounter}{1}%
    \vspace{.10in}%
    \begin{indented}%
       {\bf (#1)} %
}{\end{indented}}

\def\show#1{\ifdefempty{#1}{}{#1\\}}




%%%%%%%%%%%%%%%%% Identifying Information %%%%%%%%%%%%%%%
%%		 		For 301, we'd rather you DIDN'T tell us who you are   		%%
%%				in your homework so that we're not biased when we     		%%
%% 				grade. So, even if you fill this information in, it   			%%
%% 				will not show up in the document unless you uncomment 		%%
%%				 \show\myname and \show\myemail below                  		%%
%%%%%%%%%%%%%%%%%%%%%%%%%%%%%%%%%%%%%%%%%%%
\newcommand{\myhwname}{Homework 5}
\newcommand{\myname}{Anon}
\newcommand{\myemail}{anon@uic.edu}
%%%%%%%%%%%%%%%%%%%%%%%%%%%%%%%%%%%%%%%%%%%%%%%%%%%%%%%%%%%
\newcommand{\header}{%
\begin{center}
    {\Large \show\myhwname}
    % \show\myname
    % \show\myemail
    \today
\end{center}}
%%%%%%%%%%%%%%%%%%% Document Options %%%%%%%%%%%%%%%%%%%%%%
% \noquestionnumbers
\nopagenumbers
%%%%%%%%%%%%%%%%%%%%%%%%%%%%%%%%%%%%%%%%%%%%%%%%%%%%%%%%%%%

\begin{document}
\header

\begin{question}{Double It}
    M = "On input w, \\
    1. If the string is empty, reject. \\ 
    2. Move head until we see a "b" then mark it with a "y". Then, go back to the beginning of the string. If non found, go to step 6. \\
    3. Move head until we see an "a" then mark it with an "x". If non found, reject. \\
    4. Repeat, move head until we see an "a" then mark it with an "x". Then, go back to the beginning. If non found, reject. \\
    5. Go to step 2. \\
    6. Move head until we see an "a". If we find an "a", accept. Otherwise, reject." \\

    The TM halts because the algorithm rejects unwanted strings, so there is no possibility of an infinite "loop".
\end{question}

\begin{question}{Now Double It Again}
    \newcommand{\img}[2]{\begin{center}\includegraphics[scale=#1]{#2}\end{center}}
    >> \img{0.5}{Attachment.png}
\end{question}


\begin{question}{ALL of It, ALLways, ALL Over the Place}
    M = "On input B, \\
    1. Determine whether B is an appropriateley encoded DFA. \\
    2. Verify that for each state, it is also a final state. \\
    3. If at any point, there is a state that is not a final state, reject. \\
    4. If all states are visited and the conditions are met, accept." \\ \\
    This algorithm halts because the DFA has finite states which means the algorithm will eventually halt.
\end{question}
\end{document}
