%%%%%%%%%%%%%%%%%%%%%%%%%%%%%%%%%%%%%%%%%%%%
% LaTeX is a markup language that makes it easier to include mathematical symbols in your typed work. Some would also argue that the commands allow you to focus more on the content and less on the layout, but you can be the judge there.
%
%%%%%% TIP %%%%%%%%%%%%%%%%%%%%%%%%%%%%%%%%%
% If you are ever unsure of how to make a particular symbol, detexify (https://detexify.kirelabs.org/classify.html) allows you to draw the symbol and tells you the command.
%
%%%%%% MATH MODE %%%%%%%%%%%%%%%%%%%%%%%%%%%
% An important thing to note is that if you want to use mathematical symbols, you will need to enter math mode. There are two ways to do this. 
% 
% 1: in-line. Enclose your expression or command with dollar signs ($)
% examples:
% My favorite equation is $2+2=4$, what's yours?
% To add up a lot of numbers, we use the sum symbol: $\sum{}$
%
% 2: separate line. If you want your equation to appear on its own line, enclose it with \[ \].
% example:
% \[ x^2 + y^2 = z^2 \]
%
%%%% USEFUL COMMANDS %%%%%%%%%%%%%%%%%%%%%%%%
% 
% Easy line break: \\
% 
% The following must be used in math mode:
%   exists: \exists
%   for all: \forall
%   logical not: \neg
%   logical and: \wedge
%   logical or: \vee
%   implication: \implies
%   bi-implication: \iff
%   greater than or equal to: \geq
%   less than or equal to: \leq
%   not equal to: \neq
%
%   A note about exponents: if you only need a single character to be an exponent, you can simply use ^ to create the exponent (example: $2^3 = 8$). If you want more than one character in the exponent (for example, a 2-digit number) you can use curly braces to indicate the part to be raised: $2^{10} = 1024$
%%%%%%%%%%%%%%%%%%%%%%%%%%%%%%%%%%%%%%%%%%%%%

%% Document setup: don't need to edit this part

\documentclass[14pt]{extarticle} % supported font sizes: 8pt, 9pt, 10pt, 11pt, 12pt, 14pt, 17pt and 20pt.
\usepackage{amsmath} % allows /text{} within math mode (inserts regular text without repeatedly exiting and entering math mode)

\usepackage[margin=0.5in]{geometry} % by default, LaTeX creates very large margins. This makes a normal-sized margin.

\usepackage[shortlabels]{enumitem} % allows customized labels for enumerate (e.g. a b c instead of 1. 2. 3.)

\title{CS 151: Mathematical Foundations of Computing \\ Homework Assignment 1 \\ 
Fall 2023}
\date{}


%%%%%%%%%%%%%%%%%%%%%%%%%%%%%%%%%%%%%%%%%%%%%%%%%%%%%%%%%%%
%% Document content: this is where you put your answers

\begin{document}

\maketitle % we specified the title above, but this command actually displays it
\vspace{-0.75in} % shortens the gap between title and content

\begin{enumerate} % creates an ordinary numbered list

    % within enumerate, you must use \item to tell LaTeX where to put each number

    %% question 1 %%%%%%%%%%%%%%%%%%%%%%%%%%%
    \item
    \begin{enumerate}[a.] % creates a numbered list with labels formatted as lowercase letters with a period. This is nested within the numbered list.
        \item % a.
        Original: If I win the lottery, then I will become rich. \\
        Converse: If I become rich, then I will win the lottery. \\ 
        Inverse: If I don't win the lottery, then I will not become rich. \\
        Contrapositive: If I don't become rich, then I will not win the lottery. \\
        \item % b.
        Original: If I smell pepper, then I will sneeze. \\
        Converse: If I sneeze, then I will smell pepper. \\
        Inverse: If I don't smell pepper, then I will not sneeze. \\
        Contrapositive: If I don't sneeze, then I will not smell pepper. \\
        \item % c.
        Original: If I make an omelet, then I will need to break some eggs. \\
        Converse: If I need to break some eggs, then I will make an omelet. \\
        Inverse: If I don't make an omelet, then I won't need to break some eggs. \\
        Contrapositive: If I don't need to break some eggs, then I will not make an omelet.

    \end{enumerate} % you must always have an \end{...} for every \begin{...}

    %% question 2 %%%%%%%%%%%%%%%%%%%%%%%%%%%
    \item
    \begin{enumerate}[a.]
        \item \textbf{True} if and only if $x = 0.25$. If $x = 0.25$, then $\sqrt{0.25} = 0.5$ \\
        \textbf{False} if $x \neq 0.25$. If $x = 1$, then $\sqrt{1} = 1$ and $1 \neq 0.5$
        \item \textbf{True} if and only if $x = 0$. If $x = 0$, then $x + x = x$ because $0 + 0 = 0$ \\
        \textbf{False} if $x \neq 0$. If $x = -1$, then $-1 + -1 = -2$ and $-2 \neq -1$
        \item \textbf{True} for all integers. For all integers, x is either even or odd.
        \item \textbf{False} if $x = 0$. If $x = 0$, then $\frac{1}{0} = undefined$ \\
        \textbf{True} for all integers except 0. If $x = 100$, then $\frac{1}{100} \leq 1$. 
    \end{enumerate}
    
    %% question 3 %%%%%%%%%%%%%%%%%%%%%%%%%%%
    \item
    \begin{enumerate}[a.]
        \item $\exists x(B(x) \land \forall y((y \neq x \land B(y)) \rightarrow S(x, y)))$ 
        \item $\exists x \exists y \exists z ((B(x) \land C(y) \land C(z)) \land y \neq z) \land (S(x, y) \land S(x, z))$
        \item $\neg \exists x \exists y (C(x) \land S(x, y))$
        \item $\exists x (B(x) \land \exists y (C(y)) \land S(x, y) \land \forall z(y \neq z \rightarrow \neg S(x, z)))$ \\
    \end{enumerate}

    %% question 4 %%%%%%%%%%%%%%%%%%%%%%%%%%%
    \item
    \begin{enumerate}[a.]
        \item 
        \[\neg (p \land q) \land \neg(p \land \neg q) \text{ and } \neg p\] 
        \[\neg (p \land q) \land \neg(p \land \neg q)\]
        \[(\neg p \lor \neg q) \land (\neg p \lor q)\] 
        \[\neg p \lor (\neg q \land q) \]
        \[\neg p \lor F\]
        \[\neg p\]
        
        \item 
        \[(p \land \neg r) \rightarrow \neg q \text{ and } (p \land q) \rightarrow r\]
        \[(p \land \neg r) \rightarrow \neg q\]
        \[\neg (p \land \neg r) \lor \neg q\]
        \[\neg p \lor r \lor \neg q\]
        \[\neg p \lor \neg q \lor r\]
        \[\neg (p \land q) \lor r\]
        \[(p \land q) \rightarrow r\]
        
        \item
        \[p \leftrightarrow q \text{ and } \neg p \leftrightarrow \neg q\]
        \[p \leftrightarrow q\]
        \[(p \rightarrow q) \land (q \rightarrow p)\]
        \[(\neg p \lor q) \land (\neg q \lor p)\]
        \[(q \lor \neg p) \land (p \lor \neg q)\]
        \[ (\neg q \rightarrow \neg p) \land (\neg p \rightarrow \neg q)\]
        \[\neg p \leftrightarrow \neg q\]

        \item 
        \[\neg (\neg p \lor ((\neg p \lor q) \land \neg r)) \text{ and } p \land (q \rightarrow r)\]   
        \[\neg (\neg p \lor ((\neg p \lor q) \land \neg r))\]
        \[p \land \neg ((\neg p \lor q) \land \neg r)\]
        \[p \land (p \land \neg q \lor r)\]
        \[p \land (\neg q \lor r)\]
        \[p \land (q \rightarrow r)\] \\
    \end{enumerate}

    %% question 5 %%%%%%%%%%%%%%%%%%%%%%%%%%%
    \item
    \begin{enumerate}[a.]
        \item 
        \[p \rightarrow (q \rightarrow p)\]
        \[\neg p \lor (\neg q \lor p)\]
        \[\neg p \lor (p \lor \neg q)\]
        \[(\neg p \lor p) \lor \neg q\]
        \[T \lor \neg q\]
        \[T\]
        
        \item 
        \[((p \rightarrow r) \land (r \rightarrow q)) \rightarrow (p \rightarrow q)\]
        \[\neg ((\neg p \lor r) \land (\neg r \lor q)) \lor (\neg p \lor q)\]
        \[(\neg \neg p \land \neg r) \lor (\neg \neg r \land \neg q) \lor \neg p \lor q\]
        \[\neg p \lor (\neg \neg p \land \neg r)) \lor (\neg \neg r \land \neg q) \lor q\]
        \[((\neg p \lor \neg \neg p) \land (\neg p \lor \neg r)) \lor ((q \lor \neg \neg r) \land (q \lor \neg q))\]
        \[(T \land (\neg p \lor \neg r)) \lor ((q \lor \neg \neg r) \land T)\]
        \[(\neg p \lor \neg r) \lor (q \lor \neg \neg r)\]
        \[\neg p \lor (\neg r \lor \neg \neg r) \lor q\]
        \[\neg p \lor T \lor q\]
        \[T\]
    \end{enumerate}

    %% question 6 %%%%%%%%%%%%%%%%%%%%%%%%%%%
    \item
    \begin{enumerate}[a.]
        \item \textbf{Unsatisfiable.} If $p = q$, the second conditional is false. 
                However, if $p \neq q$, the first conditional is false. For it to be satisfiable, both need to be true.
        \item \textbf{Satifiable} when $q = T$ and $p = F$.
        \item \textbf{Unsatisfiable.} If $p = T$ and $q = T$, then it evalutes to False. \\
                if $p = F$ and $q = F$, it evaluates to False. \\
                if $p = T$ and $q = F$, it evaluates to False. \\
                if $p = F$ and $q = T$, it evaluates to False. \\
                Therefore, it is Unsatisfiable.
        \item \textbf{Satifiable} when $p = T$, $q = F$, and $r = F$.
    \end{enumerate}

\end{enumerate}

\end{document}


