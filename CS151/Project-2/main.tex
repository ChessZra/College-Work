%% Document setup: don't need to edit this part

\documentclass[14pt]{extarticle} % supported font sizes: 8pt, 9pt, 10pt, 11pt, 12pt, 14pt, 17pt and 20pt.
\usepackage{amsmath} % allows /text{} within math mode (inserts regular text without repeatedly exiting and entering math mode)

\usepackage[margin=0.5in]{geometry} % by default, LaTeX creates very large margins. This makes a normal-sized margin.

\usepackage[shortlabels]{enumitem} % allows customized labels for enumerate (e.g. a b c instead of 1. 2. 3.)

\usepackage{ amssymb } % includes \therefore

\title{CS 151: Mathematical Foundations of Computing \\ Homework Assignment 2 \\ 
Fall 2023}
\date{}

%%%%%%%%%%%%%%%%%%%%%%%%%%%%%%%%%%%%%%%%%%%%%%%%%%%%%%%%%%%
%% Document content: this is where you put your answers

\begin{document}

\maketitle % we specified the title above, but this command actually displays it.
\vspace{-0.75in} % shortens the gap between title and content

\begin{enumerate} % creates an ordinary numbered list
    % within enumerate, you must use \item to tell LaTeX where to put each number

    %% question 1 %%%%%%%%%%%%%%%%%%%%%%%%%%%
    \item
    \begin{enumerate}[a.]
        \item % a.
        % state argument form
        % NOTE: everything in the algin* block is automatically math mode.
        \begin{align*} 
          hypothesis \\
          ... \\
          hypothesis \\
          \over \therefore conclusion
        \end{align*}
        % show validity
        \begin{enumerate}[1.]
            \item statement, hypothesis 
            \item ...
            \item statement, rule (line numbers)
            \item ...
        \end{enumerate}
        
        \item % b.

        \item % c.

    \end{enumerate} % you must always have an \end{...} for every \begin{...}

    %% question 2 %%%%%%%%%%%%%%%%%%%%%%%%%%%
    \item
    \begin{enumerate}[a.]
        \item % a.

        \item % b. 

        \item % c.
        
        \item % d.

        \item % e.
        
        \item % f.
        
    \end{enumerate}
    

    %% question 3 %%%%%%%%%%%%%%%%%%%%%%%%%%%
    \item



    
\end{enumerate}

\end{document}
